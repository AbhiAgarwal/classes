\documentclass[11pt, oneside]{article}
\usepackage[]{geometry}
\geometry{letterpaper}
\usepackage{graphicx}
\usepackage{amssymb}
\usepackage[normalem]{ulem}
\let\oldemptyset\emptyset
\let\emptyset\varnothing
\newcommand{\forceindent}{\leavevmode{\parindent=1.5em\indent}}
\newcommand{\forceindentx}{\leavevmode{\parindent=2em\indent}}
\newcommand{\forceindenty}{\leavevmode{\parindent=1em\indent}}
\usepackage{amsmath}
\newcommand{\xRightarrow}[2][]{\ext@arrow 0359\Rightarrowfill@{#1}{#2}}
\makeatother
\title{Principles of Programming Languages - Homework 11}
\author{Abhi Agarwal}
\date{}

\begin{document}
\maketitle
\section{Problem 1}
\subsection*{(a)} 

\indent \par $t_0 <: t_1$: type $t_0$ can be safely substituted by values of type $t_1$.
\par (i) True: Number can safely be substituted by values of Number. Follows the SubRefl rule.
\par (ii) False.
\par (iii) True: Number can safely be substituted by values of Any. Rule SubAny.
\par (iv) True: Var to Const is permitted by rule SubObjMut, and Number to Any by rule SubAny.
\par (v) False.
\par (vi) False.
\par (vii) False.
\par (viii) True.
\par (ix) False.

\subsection*{(b)} 
\indent \par (i) (1): It will safely evaluate. It will produce a value. The value that fun(x).f would return would be 4. (2): TypeCall requires that the type of the argument in a call expression precisely matches the type of the function parameter that it is passed to. So it will not be well-typed with subtyping since that is not the case. The const x is missing the field g (which is a boolean). Also, x (declared on line 1) is not a subtype of y (in the parameter declared on line 2).
\par (ii) (1): It will safely evaluate. It will produce a value. The value that fun(x).f would return would be 3. (2): It is well-typed with subtype since x (declared on line 1) is a subtype of the parameter y (declared on line 2).
\par (iii) (1): It will safely evaluate. The function fun returns the argument that is passed into it (x), and thus accessing g will return true. (2): It is not well-typed with subtype since fun returns an object that is of type \{var f: number\}. This object does not have a field called g.
\par (iv) (1): It will safely evaluate. It will produce a value. The value would be 1. (2):	It is well-typed. Both of the branches x and y can be resolved to any common supertype \{const f: any, const g: any\} in accordance to the new TypeIf rule.

\end{document}  