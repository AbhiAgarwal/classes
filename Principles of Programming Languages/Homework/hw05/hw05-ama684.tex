\documentclass[11pt, oneside]{article}
\usepackage[landscape, margin=0.15in]{geometry}
\geometry{letterpaper}
\usepackage{graphicx}
\usepackage{amssymb}
\usepackage[normalem]{ulem}
\let\oldemptyset\emptyset
\let\emptyset\varnothing
\newcommand{\forceindent}{\leavevmode{\parindent=1.5em\indent}}
\newcommand{\forceindentx}{\leavevmode{\parindent=2em\indent}}
\newcommand{\forceindenty}{\leavevmode{\parindent=1em\indent}}
\usepackage{amsmath}
\newcommand{\xRightarrow}[2][]{\ext@arrow 0359\Rightarrowfill@{#1}{#2}}
\makeatother
\title{Principles of Programming Languages - Homework 5}
\author{Abhi Agarwal}
\date{}

\begin{document}
\maketitle
\section{Problem 1}
\subsection*{(a)}
\forceindent \par (i) 
$
\frac{
	\frac{
		\frac{
		}
		{
			\{x \rightarrow 3, y \rightarrow -2\} \vdash 3 \Downarrow 3
		} EvalVal
		\frac{
			x \in dom(\{x \rightarrow 3, y \rightarrow -2\})
		}
		{
			\{x \rightarrow 3, y \rightarrow -2\} \vdash x \Downarrow 3
		} EvalVar \thinspace \thinspace
		9 = 3 * 3
	}
	{
		\{x \rightarrow 3, y \rightarrow -2\} \vdash x * 3 \Downarrow 9
	} EvalTimes \thinspace
	\frac{
	}{
		(\{x \rightarrow 3, y \rightarrow -2\}) \vdash 2 \Downarrow 2
	} EvalVal \thinspace \thinspace
	11 = 9 + 2
	{
	}
}{
	\{x \rightarrow 3, y \rightarrow -2\} \vdash 3 * x + 2 \Downarrow 11
} EvalPlus
$

\par (ii)
$
\frac{
	\frac{
			\frac {
			}
			{
				\{x \rightarrow 3, y \rightarrow -2\} \vdash 2 \Downarrow 2
			} EvalVal
			\frac {
				y \in dom(\{x \rightarrow 3, y \rightarrow -2\})
			}
			{
				\{x \rightarrow 3, y \rightarrow -2\} \vdash y \Downarrow -2
			} EvalVar \thinspace \thinspace
			0 = 2 + -2
		}
		{
			\{x \rightarrow 3, y \rightarrow -2\} \vdash 2 + y \Downarrow 0
		} EvalPlus \thinspace
	env' = env [b \rightarrow 0] \thinspace
	\frac{
		\frac{
			b \in dom(\{x \rightarrow 3, y \rightarrow -2, b \rightarrow 0\})
		}
		{
			env' \vdash b \Downarrow 0
		} EvalVar \thinspace
		toBool(0) = false \thinspace
		\frac{
			y \in dom(\{x \rightarrow 3, y \rightarrow -2, b \rightarrow 0\})
		}
		{
			env' \vdash y \Downarrow -2
		} EvalVar
	}
	{
		\{x \rightarrow 3, y \rightarrow -2, b \rightarrow 0\} \vdash y \Downarrow -2
	} EvalIfElse \thinspace
}{
	\{x \rightarrow 3, y \rightarrow -2\} \vdash const \thinspace b = 2 + y; b ? x : y \Downarrow -2
} EvalIfThen
$

\subsection*{(b)}
%$
%\frac{
%	\frac{
%		toBool(1) = true
%	}
%	{
%		\{\emptyset\} \vdash 1 \&\& 5 \rightarrow 5
%	} DoAndTrue \thinspace \thinspace
%	v = toNum(3) + toNum(5)
%}{
%	\{\emptyset\} \vdash 3 + (1 \&\& 5) \rightarrow v
%} DoPlus
%$

%\vspace{1.5\baselineskip}

\forceindent \par (i) 3 + \underline{(1 \&\& 5)} $\xrightarrow{a}$ \underline{3 + 5} $\xrightarrow{b}$ 8
\par $a$: SearchBop2, DoAndTrue
\par $b$: DoPlus

\vspace{1.5\baselineskip}

%$
%\frac{
%	\frac {
%		v_0 = toNum(2) + toNum(1)
%	}
%	{
%		\{\emptyset\} \vdash 2 + 1 \rightarrow 3
%	} DoPlus
%	\frac {
%	}
%	{
%		\{v_0 = 3\} \vdash const \thinspace x = 3
%	} DoConstDecl
%	\frac {
%		v_1 = toNum(x) * toNum(0)
%	}
%	{
%		\{v_0 = 3, x = 3\} \vdash x * 0 \rightarrow 0
%	} DoTimes
%	\frac {
%		toBool(v_1) = false
%	}
%	{
%		\{v_0 = 3, x = 3, v_1 = 0\} \vdash ? x : x + x \rightarrow x + x
%	} DoIfElse
%	\thinspace
%	v_2 = toNum(3) + toNum(3)
%}{
%	\{v_0 = 3, x = 3, v_1 = 0\} \vdash x + x \rightarrow v_2
%} DoPlus
%$

%\vspace{1.5\baselineskip}

\par (ii) const x =
 \underline{2 + 1}; x * 0 ? x : x + x $\xrightarrow{a}$ 
 const x = 3; \underline{x} * 0 ? x : x + x $\xrightarrow{b}$
 const x = 3; \underline{3 * 0} ? x : x + x $\xrightarrow{c}$
 const x = 3; \underline{0 ? x : x + x} $\xrightarrow{d}$
 const x = 3; \underline{x} + x $\xrightarrow{e}$
 const x = 3; 3 + \underline{x} $\xrightarrow{f}$
  const x = 3; 3 + 3 $\xrightarrow{g}$
 const x = 3; \underline{3 + 3} $\xrightarrow{h}$
 6
\par $a$: SearchConstDecl1, DoPlus
\par $b$: SearchConstDecl2, DoVar
\par $c$: SearchConstDecl2, DoTimes
\par $d$: SearchConstDecl2, DoIfFalse
\par $e$: SearchConstDecl2, DoVar
\par $f$: SearchConstDecl2, DoVar
\par $g$: SearchConstDecl1, DoPlus
\par $h$: DoConstDecl

\subsection*{(c)}
\forceindent \par 
\par SearchPlus1
$
\frac{
	env \vdash e_2 \rightarrow e_2'
}
{
	env \vdash e_1 + e_2 \rightarrow e_1 + e_2'
}
$
\par SearchPlus2
$
\frac{
	env \vdash e_1 \rightarrow e_1'
}
{
	env \vdash e_1 + v_2 \rightarrow e_1' + v_2
}
$
\par DoPlus
$
\frac{
	v = toNum(v_1) + toNum(v_2)
}
{
	env \vdash v_1 + v_2 = v
}
$
\par Order for right-to-left evaluation: SearchPlus1, SearchPlus2, DoPlus

\subsection*{(d)}
\forceindent \par Big-step SOS
$
\frac{
	env \vdash e_1 \rightarrow v_1 \thinspace \thinspace env \vdash e_2 \rightarrow v_2 \thinspace \thinspace v = v_2
}
{
	env \vdash e_1, e_2 \Downarrow v
}
$

\par Small-step SOS
\par SearchSeq1
$
\frac{
	env \vdash e_1 \rightarrow e_1'
}
{
	env \vdash e_1, e_2 \rightarrow e_1', e_2
}
$
\par SearchSeq2
$
\frac{
	env \vdash e_2' \rightarrow e_2'
}
{
	env \vdash v_1, e_2 \rightarrow v_1, e_2'
}
$
\par DoReturnValue
$
\frac{
}
{
	env \vdash v_1, v_2 = v
}
$
\par Order for right-to-left evaluation: SearchSeq1, SearchSeq2, DoReturnValue

\subsection*{(e)}
\forceindent \par (i) This program evaluates to 5. During evaluation, the EvalVar is applied three times as follows:
\par The first application is the using occurrence of y in the definition of the function f on line 3. In this case, y was bound to the variable 3 in the call to f on line 4. 
\par The second application is the occurrence of x in the definition of the function g on line 2. x is bound to the value 2 in the constant declaration of 2 on line 1.
\par The third application is for the occurrence of y in the definition of the function g on line 2. This occurrence of y was bound to the value 3 in the call to g on line 3.

\par (ii) This program evaluates to 6. During evaluation, the EvalVar rule is applied four times as follows:
\par The first application is the using occurrence of y in the definition of function f on line 3. In this case, y was bound to the value 3 in the call to f on line 4.
\par The second application is the using occurrence of y in the definition of the function f on line 3. This occurrence was bound to the value 3 in the call to f on line 4.
\par The second application is the using occurrence of x in the definition of the inner function on line 2. This occurrence was bound to the value 3 in the call to g on line 3.
\par The third application is the using occurrence of y in the definition of the inner function g on line 2. This occurrence was bound to the value 3 in the call to the anonymous function that was returned from g on line 3.

\section{Problem 2}
\subsection*{(a)} $e_1 = (3 * y) + 4$

\subsection*{(b)} $e_1 = (x * y) + 4$

\subsection*{(c)} $e_2 = const \thinspace y = y; \thinspace 3 + y$

\subsection*{(d)} $e_2 = const \thinspace y = 3; x + y$

\subsection*{(e)} $e_3 = const \thinspace x = (function (z) (x(z))); x(y(2))$

\subsection*{(f)} $e_3 = const \thinspace x = (function (z) (y(x(z)))); x(y)$

\end{document}  