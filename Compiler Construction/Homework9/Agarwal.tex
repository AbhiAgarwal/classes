\documentclass[11pt, oneside]{article}   	% use "amsart" instead of "article" for AMSLaTeX format
\usepackage{geometry}                		% See geometry.pdf to learn the layout options. There are lots.
\geometry{letterpaper}                   		% ... or a4paper or a5paper or ... 
\usepackage{graphicx}				% Use pdf, png, jpg, or eps§ with pdflatex; use eps in DVI mode
\usepackage{amssymb}

\title{Homework 8}
\author{Abhi Agarwal}
\date{}



\begin{document}
\maketitle

\section{Basic Blocks}

\subsection{Question 1.1}

\par What are the basic blocks in the program? I'm using the dragon book page 527. I'm using a line separator to represent a block.
\par \noindent\rule{8cm}{0.4pt}
\par 1: i = 1
\par 2: h = 1
\par 3: goto 11
\par \noindent\rule{8cm}{0.4pt}
\par 4: s = h
\par 5: h = s + i
\par 6: i = s
\par 7: a = a - 1
\par 8: goto 11
\par \noindent\rule{8cm}{0.4pt}
\par 9: f = h
\par 10: return
\par \noindent\rule{8cm}{0.4pt}
\par 11: if a$>$0 goto 4
\par \noindent\rule{8cm}{0.4pt}
\par 12: goto 9
\par \noindent\rule{8cm}{0.4pt}

\newpage
\subsection{Question 1.2}
\par Can you simplify the code? I'm using the dragon book page 533 to help eliminate most of the jumps. The return value is $f$ so I'm going to assume this. To account for this I'm replacing $h$ with $f$ as when we return it's going to return the $f$ value, and so it helps eliminate the 9th and 10th jump. So the 9th, and 10th jump can be eliminated and changed here. We can be checking using the 11th statement, and then looping back to 4 if $a>0$, and returning if not.
\par \noindent\rule{8cm}{0.4pt}
\par 1: i = 1
\par 2: f = 1
\par 3: goto 8
\par \noindent\rule{8cm}{0.4pt}
\par 4: s = f
\par 5: f = s + i
\par 6: i = s
\par 7: a = a - 1
\par \noindent\rule{8cm}{0.4pt}
\par 8: if $a>0$ goto 4
\par \noindent\rule{8cm}{0.4pt}
\par 9: return
\par \noindent\rule{8cm}{0.4pt}

\subsection{Question 1.3}
\par What well-known function f of n does the code compute? Explain in detail how; for full points use a formal induction argument.
\par The algorithm starts off with $i = 1, f = 1$, and then in the while loop the condition is $a > 0$. In the body we set s to f, $f = s + i$, and then set $i = s$, and then we want to decrement a with 1. This particular function computes the nth/ath fibonacci number in the fibonacci sequence - $F_n = F_n-1 + F_n-2$. The same thing occurs in this case, but we set the previous value to be i.
\par When we decide it in that situation we can try consider it, and understand that it would work in the cases that have been provided. The last f is the ath or nth Fibonacci digit. 

\end{document}  