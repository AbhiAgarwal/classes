\documentclass[11pt, oneside]{article}
\usepackage{geometry}
\geometry{letterpaper}
\usepackage{graphicx}
\usepackage{amssymb}
\newcommand{\forceindent}{\leavevmode{\parindent=1.5em\indent}}
\newcommand{\forceindentx}{\leavevmode{\parindent=2em\indent}}
\newcommand{\forceindenty}{\leavevmode{\parindent=1em\indent}}
\title{Principles of Programming Languages - Homework 1}
\author{Abhi Agarwal}
\date{}

\begin{document}
\maketitle
\section{Problem 1}

\subsection*{(a)}
\forceindent \textbf{The use of pi at line 4 is bound at which line?} Line 3
\par \textbf{The use of pi at line 7 is bound at which line?} Line 1

\subsection*{(b)}
\forceindent \textbf{The use of x at line 3 is bound at which line?} Line 2
\par \textbf{The use of x at line 6 is bound at which line?} Line 5
\par \textbf{The use of x at line 10 is bound at which line?} Line 5
\par \textbf{The use of x at line 13 is bound at which line?} Line 1

\section{Problem 2}
\forceindent Is the body of g well-typed? Yup!
\par \textbf{If so, give the return type of g and explain how you determined this type.} The return type of g would be (Int, Int).
\par \textbf{For this explanation, first, give the types for the names a and b.}
\par Type of a is an Int. The reasoning stands from:

\par val (a, b) = (1, (x, 3))
\par \forceindentx a: Int
\par \forceindentx 1: Int
\par \forceindentx val a = 1
\par \forceindentx val \_ = 1 $=> Int$

\par One portion of the return could either be:

\par a + 2 : Int because
\par  \forceindentx a : Int
\par \forceindentx 2 : Int
\par \forceindentx $\_ + \_: (Int, Int) => Int$

\par Or

\par \forceindentx 1 : Int

\par Type of b is a Tuple of (Int, Int). The reasoning stands from:

\par val (a, b) = (1, (x, 3))
\par \forceindentx b: (Int, Int)
\par \forceindentx x: Int
\par \forceindentx 3: Int
\par \forceindentx (x, 3): (Int, Int)
\par \forceindentx val b = (x, 3) $=> (Int, Int)$

\section{Problem 3}

\subsection*{(d) ii.}
\forceindent No mutable variables? Yup!
\par Is your implementation tail-recursive? Yes it is!

\end{document}